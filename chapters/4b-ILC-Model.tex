% !TEX root = ..\Main.tex
\chapter{The Ideal Linear Commitment Model and Other Models}
\label{chapterlabel:Models}

\section{Formal Model}
\label{formalILCmodel}

In this section, we will define Ideal Linear Commitment protocols more precisely. We will then compare and contrast this model with other models.
\begin{itemize}
\item Interactive Oracle Proofs (Ben-Sasson, Chiesa, Spooner)
\item Linear Interactive Proofs and Linear PCPs (Nir Bitansky, Alessandro Chiesa, Yuval Ishai, Rafail Ostrovsky, Omer Paneth)
\item (Fully) Linear Interactive Oracle Proofs (Dan Boneh, Elette Boyle, Henry Corrigan-Gibbs, Niv Gilboa, Yuval Ishai)
\end{itemize}

\subsection{Definitions}

We begin by defining general Ideal Linear Commitment protocols, and then specialise to the case of public-coin protocols.

In a $\mu$-round Ideal Linear Commitment protocol, a prover $\PoILC$ and a verifier $\VILC$ will interact with each other via the \ILC, each sending $\mu$ messages. An Ideal Linear Commitment protocol is defined over a field $\F$ and will use a fixed vector length $\sizevect$. Let $(t_1,\ldots,t_\mu) \in \N^\mu$ be a tuple of message lengths. Let the initial state of $\PoILC$ be $\mathsf{state}^\Po_0$. Let the initial state of $\VILC$ be $\mathsf{state}^\V_0$. Let $\rho_\Po$ and $\rho_\V$ be the randomness for the prover and verifier respectively. Set $m_0 = \bot$ and $\Pi_0 = \bot$. In every round $\roundnum \in \{ 1,\ldots,\mu \}$:
\begin{itemize}
\item The prover takes as input the round number $\roundnum$, the verifier's previous messages $m_0,\ldots,m_{\roundnum-1}$, internal state $\mathsf{state}^\Po_{i-1}$, and randomness $\rho_\Po$, and outputs a matrix $V_i \in \F^{t_i \times \sizevect}$ and a new internal state $\mathsf{state}^\Po_{i}$. It commits to $V_i$ using the \ILCcommit command.
\item After round $\roundnum$, the \ILC contains the vertical concatenation of the matrices $V_1,\ldots,V_\roundnum$.
\item The verifier takes as input the round number $\roundnum$, internal state $\mathsf{state}^\V_{i-1}$, and randomness $\rho_\Po$. It makes linear \ILCopen and \ILCcheck queries on the contents of the $\ILC$. Then it computes a message \red{Fill in}
\end{itemize}

The \ILC could be viewed in several possible different ways, as a commitment functionality, communication channel, or a trusted third party. When Ideal Linear Commitment protocols are compiled into real zero-knowledge protocols, the functionality previously guaranteed by the \ILC will be enforced using various cryptographic tools.

\section{Zero-Knowledge Proofs}
\label{shvzkdef}

A \emph{proof system} is defined by a triple of stateful PPT algorithms $(\mathcal{K},\mathcal{P},\mathcal{V})$, which we call the common-reference-string \emph{generator}, the \emph{prover} and \emph{verifier}, respectively.

The setup generator $\mathcal{K}$ creates public parameters $\crs$ which provide the necessary setup information for \prover and \verifier to run the protocol. On input $1^\lambda$, the generator $\crsgen$ produces a common reference string $\crs$. We think of $\crs$ as being honestly generated and use the public parameter model purely for simplicity and efficiency in our proofs. However, in the proofs we construct, $\crs$ consists of parts that are either publicly verifiable or could be generated by the verifier, so we do not rely on the public parameter model for security in any way.

The prover and verifier communicate with each other through a \emph{communication channel} $\overset{\textnormal{chan}}{\longleftrightarrow}$. When $\mathcal{P}$ and $\mathcal{V}$ interact on inputs $s$ and $t$ through a communication channel $\overset{\textnormal{chan}}{\longleftrightarrow}$ we let $\viewV \leftarrow \langle \mathcal{P}(s) \overset{\textnormal{chan}}{\longleftrightarrow} \mathcal{V}(t)\rangle$ be the view of the verifier in the execution, which is made up of all of the verifier's inputs, including random coins, and let $\viewp \leftarrow  \langle \mathcal{P}(s) \overset{\textnormal{chan}}{\longleftrightarrow} \mathcal{V}(t)\rangle$ denote the transcript of the communication between prover and channel. This overloads the notation $\leftarrow  \langle \mathcal{P}(s) \overset{\textnormal{chan}}{\longleftrightarrow} \mathcal{V}(t)\rangle$ but it will always be clear from the variable name if we get the verifier's view or the prover's transcript. At the end of the interaction the verifier accepts or rejects. We write $\langle \mathcal{P}(s)\overset{\textnormal{chan}}{\longleftrightarrow} \mathcal{V}(t)\rangle =b$ depending on whether the verifier rejects ($b=0$) or accepts ($b=1$).

In the \emph{standard channel} $\std$, all messages are forwarded between prover and verifier. 
We also consider an \emph{ideal linear commitment}  channel, \ILC, described in Figure~\ref{ILCSyntaxFigure11}. 
When using the \ILC\ channel, the prover can submit a \ILCcommit\ command to commit to 
vectors of field elements of some fixed length $\sizevect$, specified in $\crs_{\ILC}$. The vectors remain secretly stored in the channel, and will not be forwarded to the verifier. Instead, the verifier only learns how many vectors the prover has committed to, and their lengths. The verifier can submit a \ILCsend\ command to the \ILC\ to send field elements to the prover. In addition, the verifier can also submit \ILCopen\ queries to the \ILC\ for obtaining the opening of any linear combinations of the vectors \emph{of the same length} sent by the prover. We stress that the verifier can request several linear combinations within a single \ILCopen\ query, as depicted in Figure~\ref{ILCSyntaxFigure11}.

In addition to the \ILC\ commands used in the original model \cite{BootleCGGHJ17}, we introduce the next command \ILCcheck. Inside the \ILC \ model, this command behaves in an identical way to the \ILCsend\ command. However, the two commands will be treated slightly differently when \ILC\ protocols are compiled into real protocols. The reason for making the distinction is that in some of our protocols, the verifier needs to check whether a large vector, that they have computed themselves, is the correct linear combination of vectors committed by the prover. This could be solved by having the verifier make an \ILCsend\ query for the correct linear combination and checking whether the result is equal to the large vector, which incurs a communication cost for the large vector in the \ILC\ protocol. However, in a real protocol, where the verifier's queries will actually be computed and sent by the prover, the verifier can re-commit to the vector that they have computed, and check against the prover's commitments. In other words, since the verifier has already computed the vector for themself, there is no need for them to receive it again. Therefore, the \ILCcheck\ command is used to distinguish this case, which should not be counted as part of the communication costs of a proof. This issue will be discussed further in later chapters.

\begin{figure}[htb]
\resizebox{\textwidth}{!}{\begin{tabular}{lll}
$\Po_{\ILC~}$ &\multirow{4}{*}{ \resizebox{0.8\linewidth}{!}{\includegraphics{ILC2-shrink.png}} }&$\V_{\ILC~}$\\
&& \\&&\\&&\\&&\\&&\\&&\\&&\\&&\\&&\\&&\\&&\\
&&\\&&\\
\end{tabular}}\caption{Description of the $\ILC~$ channel.}\label{ILCSyntaxFigure11}
\end{figure}

In a proof system over the \ILC\ channel, sequences of \ILCcommit, \ILCsend\ and \ILCopen\ and \ILCcheck\ queries could alternate in an arbitrary order. We call a proof system over the \ILC\ channel \emph{non-adaptive} if the verifier makes one \emph{open} query and then immediately makes one \emph{check} query to the \ILC\ channel, before terminating his interaction with the channel, and these are the only \emph{open} and \emph{check} queries that the verifier makes. Otherwise we call it \emph{adaptive}. Although adaptive proof systems are allowed by the model, in this paper we will only consider non-adaptive \ILC\ proof systems to simplify the exposition. In non-adaptive \ILC\ proof systems with vectors of length $\sizevect$, the verifier will produce two query matrices, $Q$ for the \ILCopen\ query and $Q$ for the \ILCcheck\ query. 

In fact, in our protocols, we will allow the prover to commit to vectors of \emph{several different fixed lengths} $\sizevect_1,\ldots,\sizevect_r$, which will be specified at the beginning of the protocol. This is easily formalised by defining a new communication channel, giving the prover and verifier access to $r$ copies of different \ILC\ channels of lengths $\sizevect_1,\ldots,\sizevect_r$. For the new channel, the \ILCcommit\ command on vectors $\vec{v}_1,\ldots,\vec{v}_m$ separates the vectors into sets of vectors of lengths $\sizevect_1,\ldots,\sizevect_r$, and commits to the set of vectors of length $\sizevect_i$ using the $i$th \ILC\ channel on vectors of length $\sizevect_i$. The \ILCsend\ command for the new channel simply takes the verifier's message and sends it to each of the $r$ \ILC\ channels using their own \ILCsend\ command. For the \ILCopen\ and \ILCcheck\ commands for the new channel, the verifier's queries take the form $Q = (Q_1,\ldots,Q_r)$, where each $Q_i$ is a query matrix for the $i$th \ILC\ channel. The new channel returns the responses of all of the \ILC\ queries to the verifier. Such a channel will be referred to as an \ILC\ channel for vectors of several fixed lengths, and will be specified simply by including multiple vector lengths in $\crs_{\ILC~}$ rather than just one.

Alternatively, we can easily incorporate vectors of different lengths into the model by padding all vectors with zeroes until they are the same length as the longest vector. This will have no impact on the asymptotic efficiency of our protocols. Some of our \ILC\ protocols require single values to be committed as well as vectors. 

We remind the reader that \ILC\ proof systems are different from linear interactive proofs considered in \cite{BitanskyCIPO13}. In linear interactive proofs both the prover and verifier send vectors of field elements, but the prover can only send linear (or affine) transformations of the verifier's previously sent vectors. However, for our constructions it is important that the prover can compute on field elements received by the verifier and for instance evaluate polynomials.

We say a proof system is \emph{public coin} if the verifier's messages to the communication channel are chosen uniformly at random and independently of the actions of the prover, i.e., the verifier's messages to the prover correspond to the verifier's randomness $\rho$.

We will model our zero-knowledge proofs using ternary relations $\R$ consisting of tuples $(\crs, \stm,\wit)$. The first item in the tuple is the \emph{common reference string} $\crs$ containing the setup information required for the protocol. Typically, $\crs$ will specify the security parameter $\sep$, perhaps implicitly through its length, and may also contain other parameters used for specifying the specific relation, e.g. a description of a field. Often, $\crs$ will also contain parameters that do not influence membership of $\R$ but may aid the prover and verifier, for instance, a description of an encoding function that they will use. The second item in the tuple, $\stm$ is the \emph{instance} and represents what the prover wants to prove. The final item, $\wit$, is the prover's secret \emph{witness} that $(\crs,\stm)\in\LL_\R$ where the language $\LL_\R \subset \{0,1\}^*$ as follows.
\[ \LL_\R=\{(\crs,\stm)|\exists \wit: (\crs,\stm,\wit)\in \R\} \]
Intuitively speaking, this is the collection of statements which are `true', and for which the verifier should output one after running the protocol with an honest prover. The languages $\LL_\R$ are decidable in polynomial time.

The protocol $(\KK,\mathcal{P},\mathcal{V})$ is called a \emph{proof of knowledge} over communication channel $\overset{\textnormal{chan}}{\longleftrightarrow}$ for relation $\R$ if it has perfect completeness and computational knowledge soundness as defined below.

\begin{definition}[Perfect Completeness]
The proof is \emph{perfectly complete} if for all PPT adversaries $\mathcal{A}$
$$\Pr \left[ \begin{array}{c} \crs\gets \KK(1^\sep); (u,w) \leftarrow \mathcal{A}(\crs): \\
(\crs,u,w)\notin \R ~\vee~ \langle \mathcal{P}(\crs,u,w)\overset{\textnormal{chan}}{\longleftrightarrow}\mathcal{V}(\crs,u)\rangle=1 \end{array}\right] =1.$$
\end{definition}

\begin{definition}[Knowledge soundness]
A public-coin proof system has \emph{computational (strong black-box) knowledge soundness} if for all DPT $\Po^*$ there exists an expected PPT extractor $\E$ such that for all PPT adversaries $\A$
$$\Pr\left[\begin{array}{c} \crs\gets \KK(1^\sep);(\stm,s) \gets \mathcal{A}(\crs); \wit\gets \E^{\langle \Po^*(s)\overset{\textnormal{chan}}{\longleftrightarrow}\V(\crs,\stm)\rangle}(\crs,\stm):\\ b=1 ~\wedge~ (\crs,\stm,\wit)\notin \R \end{array}\right]\approx 0.$$ 
Here the oracle $\langle \Po^*(s)\overset{\textnormal{chan}}{\longleftrightarrow}\V(\crs,\stm)\rangle$ runs a full protocol execution and if the proof is successful it returns a transcript of the prover's communication with the channel. The extractor $\E$ can ask the oracle to rewind the proof to any point in a previous transcript and execute the proof again from this point on with fresh public-coin challenges from the verifier. We define $b\in \{0,1\}$ to be the verifier's output in the first oracle execution, i.e., whether it accepts or not, and we think of $s$ as the state of the prover. The definition can then be paraphrased as saying that if the prover in state $s$ makes a convincing proof, then we can extract a witness.

If the definition holds also for unbounded $\mathcal{P}^*$ and $\A$ we say the proof has {\em statistical knowledge soundness}. 

If the definition of knowledge soundness holds for a non-rewinding extractor, i.e., a single transcript of the prover's communication with the communication channel suffices, we say the proof system has knowledge soundness with {\em straight-line extraction}. 
\end{definition}
\noindent
This definition gives a security guarantee against computationally bounded adversaries. Zero-knowledge protocols satisfying this definition are properly called zero-knowledge arguments of knowledge to distinguish them from zero-knowledge proofs of knowledge, for which knowledge soundness is guaranteed even against unbounded adversaries. However, the term `proofs' is often used in both cases.

Another way to define a proof of knowledge follows Groth and Ishai~\cite{GrI08} who borrowed the term witness-extended emulation from Lindell~\cite{Lin03}. Informally, their definition says that given an adversary that produces an acceptable argument with some probability, there exists an emulator that produces a similar argument with the same probability together with a witness $w$. Note that the emulator is allowed to rewind the prover and verifier's interaction to any previous move.

\begin{definition}[Witness-extended emulation]
$(\mathcal{P},\mathcal{V})$ has {\em statistical witness-extended emulation} if for all deterministic polynomial time $\mathcal{P}^*$ there exists an expected polynomial time emulator $\mathcal{E}$ such that for all interactive adversaries $\mathcal{A}$
\begin{align*}
& \Pr \Big{[}(u,s) \leftarrow\mathcal{A}(1^\sep); tr \leftarrow \langle \mathcal{P}^*(u,s), \mathcal{V}(u)\rangle: \mathcal{A}(tr)=1 \Big{]}\\
\approx \ & \Pr \left[ \begin{array}{l} (u,s) \leftarrow \mathcal{A}(1^\sep); (tr,w) \leftarrow \mathcal{E}^{ \langle \mathcal{P}^*(u,s), \mathcal{V}(u) \rangle } (u): \\ \mathcal{A}(tr)=1 \text{\emph{ and if }} tr \text{\emph{ is accepting then }} (u,w) \in R \end{array} \right]
\end{align*}
where the oracle called by $\mathcal{E}^{\langle \mathcal{P}^*(u,s), \mathcal{V}(u)\rangle}$ permits rewinding to a specific point and resuming with fresh randomness for the verifier from this point onwards.
\end{definition}
\noindent
Note that in this definition, the role of the adversary $\mathcal{A}$ is as a distinguisher, while it is the job of $\mathcal{P}^*$ to produce proofs. That is why the two are decoupled from one another.

We can interpret $s$ as the state of $\mathcal{P}^*$, including the randomness. So, whenever $\mathcal{P}^*$ is able to make a convincing argument when in state $s$,  $\mathcal{E}$ can extract a witness. Witness Extended Emulation implies Knowledge Soundness \cite{dissertation}.

We will construct public-coin proofs that have special honest-verifier zero-knowledge. This means that if the verifier's challenges are known, or even adversarially chosen, then it is possible to simulate the verifier's view without the witness.  In other words, the simulator works for verifiers who may use adversarial coins in choosing their challenges but they follow the specification of the protocol as an honest verifier would. 
\begin{definition}[Special Honest-Verifier Zero-Knowledge]
The proof of knowledge is \emph{computationally special honest-verifier zero-knowledge (SHVZK)} if there exists a PPT simulator $\mathcal{S}$ such that for all stateful interactive PPT adversaries $\mathcal{A}$ that output $(u,w)$ such that $(\crs,u,w)\in R$ and randomness $\rho$ for the verifier
\begin{eqnarray*}
&&\Pr \left[ \begin{array}{c}\crs\gets \KK(1^\sep);(u, w, \rho) \leftarrow\mathcal{A}(\crs); \\
\viewV\leftarrow \langle \mathcal{P}(\crs,u,w)\overset{\textnormal{chan}}{\longleftrightarrow} \V(\crs,u;\rho)\rangle: \A(\viewV)=1\end{array} \right]\\
&\approx &\Pr \left[ \crs\gets \KK(1^\sep);(u, w, \rho) \leftarrow\mathcal{A}(\crs); \viewV\leftarrow \mathcal{S}(\crs,u,\rho): \mathcal{A}(\viewV)=1\right].
\end{eqnarray*}

We say the proof is \emph{statistically SHVZK} if the definition holds also against unbounded adversaries, and we say the proof is \emph{perfect SHVZK} if the probabilities are exactly equal.
\end{definition}