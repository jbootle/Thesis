% -------- Packages --------

% This package just gives you a quick way to dump in some sample text.
% You can remove it -- it's just here for the examples.
\usepackage{blindtext}

% This package means empty pages (pages with no text) won't get stuff
%  like chapter names at the top of the page. It's mostly cosmetic.
\usepackage{emptypage}

% The graphicx package adds the \includegraphics command,
%  which is your basic command for adding a picture.
\usepackage{graphicx}
% Use this command to keep your images in a separate
%  'images' folder.
\graphicspath{ {images/} }

% The float package improves LaTeX's handling of floats,
%  and also adds the option to *force* LaTeX to put the float
%  HERE, with the [H] option to the float environment.
\usepackage{float}

% The amsmath package enhances the various ways of including
%  maths, including adding the align environment for aligned
%  equations.
\usepackage{amsmath}
\usepackage{amsfonts}

% Use these two packages together -- they define symbols
%  for e.g. units that you can use in both text and math mode.
\usepackage{gensymb}
\usepackage{textcomp}
% You may also want the units package for making little
%  fractions for unit specifications.
%\usepackage{units}


% The setspace package lets you use 1.5-sized or double line spacing.
\usepackage{setspace}
\setstretch{1.5}

% That just does body text -- if you want to expand *everything*,
%  including footnotes and tables, use this instead:
%\renewcommand{\baselinestretch}{1.5}


% PGFPlots is either a really clunky or really good way to add graphs
%  into your document, depending on your point of view.
% There's waaaaay too much information on using this to cover here,
%  so, you might want to start here:
%   http://pgfplots.sourceforge.net/
%  or here:
%   http://pgfplots.sourceforge.net/pgfplots.pdf
%\usepackage{pgfplots}
%\pgfplotsset{compat=1.3} % <- this fixed axis labels in the version I was using

% PGFPlotsTable can help you make tables a little more easily than
%  usual in LaTeX.
% If you're going to have to paste data in a lot, I'd suggest using it.
%  You might want to start with the manual, here:
%  http://pgfplots.sourceforge.net/pgfplotstable.pdf
%\usepackage{pgfplotstable}

% These settings are also recommended for using with pgfplotstable.
%\pgfplotstableset{
%	% these columns/<colname>/.style={<options>} things define a style
%	% which applies to <colname> only.
%	empty cells with={--}, % replace empty cells with '--'
%	every head row/.style={before row=\toprule,after row=\midrule},
%	every last row/.style={after row=\bottomrule}
%}

\usepackage{pdflscape}
\usepackage{afterpage}
\usepackage{capt-of}

% The mhchem package provides chemistry formula typesetting commands
%  e.g. \ce{H2O}
%\usepackage[version=3]{mhchem}

% And the chemfig package gives a weird command for adding Lewis 
%  diagrams, for e.g. organic molecules
%\usepackage{chemfig}

% The linenumbers command from the lineno package adds line numbers
%  alongside your text that can be useful for discussing edits 
%  in drafts.
% Remove or comment out the command for proper versions.
%\usepackage[modulo]{lineno}
% \linenumbers 


% Alternatively, you can use the ifdraft package to let you add
%  commands that will only be used in draft versions
%\usepackage{ifdraft}

% For example, the following adds a watermark if the draft mode is on.
%\ifdraft{
%  \usepackage{draftwatermark}
%  \SetWatermarkText{\shortstack{\textsc{Draft Mode}\\ \strut \\ \strut \\ \strut}}
%  \SetWatermarkScale{0.5}
%  \SetWatermarkAngle{90}
%}


% The multirow package adds the option to make cells span 
%  rows in tables.
\usepackage{multirow}


% Subfig allows you to create figures within figures, to, for example,
%  make a single figure with 4 individually labeled and referenceable
%  sub-figures.
% It's quite fiddly to use, so check the documentation.
%\usepackage{subfig}

% The natbib package allows book-type citations commonly used in
%  longer works, and less commonly in science articles (IME).
% e.g. (Saucer et al., 1993) rather than [1]
% More details are here: http://merkel.zoneo.net/Latex/natbib.php
%\usepackage{natbib}

% The bibentry package (along with the \nobibliography* command)
%  allows putting full reference lines inline.
%  See: 
%   http://tex.stackexchange.com/questions/2905/how-can-i-list-references-from-bibtex-file-in-line-with-commentary
\usepackage{bibentry} 

% The isorot package allows you to put things sideways 
%  (or indeed, at any angle) on a page.
% This can be useful for wide graphs or other figures.
%\usepackage{isorot}

% The caption package adds more options for caption formatting.
% This set-up makes hanging labels, makes the caption text smaller
%  than the body text, and makes the label bold.
% Highly recommended.
\usepackage[format=hang,font=small,labelfont=bf]{caption}

% If you're getting into defining your own commands, you might want
%  to check out the etoolbox package -- it defines a few commands
%  that can make it easier to make commands robust.
\usepackage{etoolbox}

\usepackage{algorithm,algorithmicx}

\usepackage{amssymb}

\usepackage{tikz}

\usepackage{amsxtra}
\usepackage{bm}
\usepackage{mathrsfs} 

\usepackage{paralist}
\usepackage{graphicx}
\usepackage{url} 
%\usepackage[usenames,dvipsnames]{xcolor}
\usetikzlibrary{positioning}
\usepackage{multirow}
\usepackage{breakcites}
\usepackage{microtype}
\usepackage{color}
%\usepackage{hyperref}
\usepackage{epstopdf}

\newtheorem{defn}{Definition}
\newtheorem{definition}{Definition}
\newtheorem{thm}{Theorem}
\newtheorem{theorem}{Theorem}
\newtheorem{claim}{Claim}
\newtheorem{cor}[thm]{Corollary}
\newtheorem{lemma}[thm]{Lemma}
\newtheorem{prp}{Proposition}

%==========================Commands===========================

%===Editorial========

\newcommand{\red}[1]{{\textcolor{red}{#1}}}

%===Computational Complexity========

\newcommand{\sep}{\ensuremath{\lambda}}
\newcommand{\bigo}{\ensuremath{\mathcal{O}}}
\newcommand{\instancegen}{\ensuremath{\mathcal{G}}}
\newcommand{\instance}{\ensuremath{\mathsf{gk}}}

%===Relations and Languages========

\newcommand{\relation}{\ensuremath{\mathsf{R}}}
\newcommand{\lang}{\ensuremath{\mathsf{L}}}
\newcommand{\statement}{\ensuremath{\mathsf{u}}}
\newcommand{\witness}{\ensuremath{\mathsf{w}}}

%===Pseudocode========
\newcommand{\Function}[2]{\textbf{#1}({#2})}
\newcommand{\If}{\textbf{If }}
\newcommand{\Else}{\textbf{Else }}
\newcommand{\While}{\textbf{While }}
\newcommand{\Return}{\textbf{Return }}

%===Zero-Knowledge========

%=Algorithms

\newcommand{\crsgen}{\ensuremath{\mathcal{G}}}
\newcommand{\prover}{\ensuremath{\mathcal{P}}}
\newcommand{\verifier}{\ensuremath{\mathcal{V}}}
\newcommand{\extractor}{\ensuremath{\mathcal{X}}}
\newcommand{\simulator}{\ensuremath{\mathcal{S}}}
\newcommand{\adversary}{\ensuremath{\mathcal{A}}}

%=Objects
\newcommand{\crs}{\ensuremath{\mathsf{\sigma}}}
\newcommand{\Proof}{\ensuremath{\mathsf{\pi}}}
\newcommand{\transcript}{\ensuremath{\mathsf{tr}}}
\newcommand{\provstate}{\ensuremath{\mathsf{s}}}
\newcommand{\verifstate}{\ensuremath{\mathsf{t}}}

%===Commitment Schemes========
\newcommand{\ck}{\ensuremath{\mathsf{ck}}}
\newcommand{\mlen}{\ensuremath{n}}
\newcommand{\ckgen}[2]{\textbf{KeyGen} \left( #1 , #2 \right)}
\newcommand{\commit}[2]{\textbf{Commit}_\ck \left( #1 ; #2 \right)}
\newcommand{\decommit}[2]{\textbf{Decommit}_\ck \left( #1 ; #2 \right)}
\newcommand{\verify}[2]{\textbf{Verify}_\ck \left( #1 ; #2 \right)}

%===Maths========

\newcommand{\N}{\ensuremath{\mathbb{N}}}
\newcommand{\Z}{\ensuremath{\mathbb{Z}}}
\newcommand{\F}{\ensuremath{\mathbb{F}}}
\newcommand{\G}{\ensuremath{\mathbb{G}}}
\newcommand{\R}{\ensuremath{\mathcal{R}}}
\renewcommand{\vec}[1]{\mathbf{#1}}
\newcommand\norm[2]{\ensuremath{\left\lVert#2\right\rVert_{#1}}}
\newcommand\abs[1]{|#1|}
\newcommand*{\qed}{\hfill\ensuremath{\blacksquare}}%
\newcommand*{\qeda}{\hfill\ensuremath{\square}}%

%=DLOG Parameters

\newcommand{\DLOGprime}{\ensuremath{p}}

%=Lattice Cryptography

\newcommand{\SISmodulus}{\ensuremath{q}}
\newcommand{\SISnrows}{\ensuremath{n}}
\newcommand{\SISncols}{\ensuremath{m}}
\newcommand{\SISnormbound}{\ensuremath{\beta}}

%=SRSA Parameters

\newcommand{\RSAmodulus}{\ensuremath{N}}
\newcommand{\RSAexponent}{\ensuremath{e}}
\newcommand{\RSAp}{\ensuremath{p}}
\newcommand{\RSAq}{\ensuremath{q}}
\newcommand{\RSAC}{\ensuremath{C}}
\newcommand{\RSAM}{\ensuremath{M}}

%=CRHF Parameters

\newcommand{\HashFunction}{\ensuremath{h}}
\newcommand{\HashInLen}{\ensuremath{m}}
\newcommand{\HashOutLen}{\ensuremath{l}}

%===ILC Model========

\newcommand{\ilc}{\ensuremath{\overset{\mathsf{ILC}}{\longleftrightarrow}}}
\newcommand{\ILC}{\ensuremath{\mathsf{ILC}}}
\newcommand{\ILCcommit}{\texttt{commit}}
\newcommand{\ILCsend}{\texttt{send}}
\newcommand{\ILCopen}{\texttt{open}}
\newcommand{\ILCcheck}{\texttt{check}}
\newcommand{\ILCveclen}{n}

%===Arithmetic Circuits========
\newcommand{\ACnumgates}{N}
\newcommand{\ACveclen}{n}
\newcommand{\ACnumvec}{m}
\newcommand{\ACprime}{p}

%===Non-ILC Optimisations========
\newcommand{\extdeg}{k}

%========================Miscellaneous Commands to Sort========================
\newcommand{\Gr}{\G}
\newcommand{\Grp}{\mathbb{G}}
\newcommand{\order}{\mathbb{p}}
\newcommand{\negl}{\nu}
\newcommand{\prim}[2]{#1.#2}    

\newcommand{\zo}{\ensuremath{\{0,1\}}}

\newcommand{\mat}[1]{\mathbf{#1}}
\newcommand{\COMMENT}[1]{}


%\makeatletter
%\DeclarePairedDelimiter{\ceil}{\lceil}{\rceil}
%\DeclarePairedDelimiter{\floor}{\lfloor}{\rfloor}

\let\CapLet\MakeUppercase
\let\LowLet\MakeLowercase

% Prover, Verifier, Statements 
\newcommand{\Po}{\ensuremath{\mathcal{P}}}
\newcommand{\prov}{\Po}
\newcommand{\V}{\ensuremath{\mathcal{V}}}
\newcommand{\ver}{\V}
\newcommand{\LL}{\ensuremath{\mathcal{L}}}
\newcommand{\Rac}{\ensuremath{\mathcal{R}_{\mathsf{AC}}}}
\newcommand{\KK}{\ensuremath{\mathcal{K}}}
\newcommand{\KKIOP}{\ensuremath{\mathcal{K}_{\IOP}}}
\newcommand{\KKILC}{\ensuremath{\mathcal{K}_{\ILC}}}
\newcommand{\C}{\ensuremath{\mathcal{C}}}
\newcommand{\E}{\ensuremath{\mathcal{E}}}
\newcommand{\Si}{\ensuremath{\mathcal{S}}}
\newcommand{\iop}{\ensuremath{\overset{\mathsf{IOP}}{\longleftrightarrow}}}
\newcommand{\schan}{\std}%merging two notations
\newcommand{\std}{\ensuremath{\longleftrightarrow}}
\newcommand{\IOP}{\ensuremath{\mathsf{IOP}}}

\newcommand{\vereq}{\overset{?}{=}}



\newcommand{\Ex}{\ensuremath{\mathcal{E}}}
\newcommand{\ExILC}{\ensuremath{\mathcal{E}_{\ILC}}}
\newcommand{\ExIOP}{\ensuremath{\mathcal{E}_{\IOP}}}
\newcommand{\Sim}{\ensuremath{\mathcal{S}}}
\newcommand{\stm}{u}
\newcommand{\wit}{w}


\newcommand{\PoMal}{\Po^*}
\newcommand{\PoILC}{\ensuremath{\mathcal{P}_{\ILC}}}
\newcommand{\PoILCMal}{\PoILC^*}
\newcommand{\PoIOP}{\ensuremath{\mathcal{P}_{\IOP}}}
\newcommand{\PoIOPMal}{\PoIOP^*}
% \newcommand{\PoACMal}{\Po^*_{AC}}
\newcommand{\PoShiftMal}{\Po^*_{\textnormal{shift}}}
\newcommand{\PoProdMal}{\Po^*_{\textnormal{prod}}}
\newcommand{\VMal}{\V^*}
\newcommand{\VILC}{\ensuremath{\mathcal{V}_{\ILC}}}
\newcommand{\VIOP}{\ensuremath{\mathcal{V}_{\IOP}}}
\newcommand{\SimILC}{\ensuremath{\mathcal{S}_{\ILC}}}
\newcommand{\SimIOP}{\ensuremath{\mathcal{S}_{\IOP}}}


\newcommand{\Emal}{E^*}
\newcommand{\rownr}{\tau}
\newcommand{\numcolt}{\numcol}% I use these commands to merge two notations I think are the same, but not confident enough to do replace all on
\newcommand{\EC}{E_C}
\newcommand{\witsim}{\widetilde{\wit}}
\newcommand{\Extrans}{\ensuremath{\mathcal{T}}}
\newcommand{\viewpilc}{\mathsf{trans}_{\PoILC}}
\newcommand{\viewp}{\mathsf{trans}_{\Po}}
\newcommand{\viewpilcsim}{\widetilde{\viewpilc}}



\newcommand{\ComSetup}{\mathsf{Setup}}
\newcommand{\ComCommit}{\mathsf{Commit}}
\newcommand{\ComOpen}{\mathsf{Open}}
\newcommand{\ComVerify}{\mathsf{Verify}}

\newcommand{\setI}{\mathcal{I}}
\newcommand{\pp}{\mathsf{pp}} % public parameters
\newcommand{\vecsize}{\ell} % row size
\newcommand{\elmvecsize}{\rho} % column size
\newcommand{\Co}{c}  %commitment%
\newcommand{\aux}{\text{aux}} % auxilary info for commitment etc.
\newcommand{\asn}{\leftarrow} % assignment  operator
\newcommand{\A}{\mathcal{A}} % Adversary
\newcommand{\AIOP}{\mathcal{A}_{\IOP}} % Adversary
\newcommand{\AILC}{\mathcal{A}_{\ILC}} % Adversary
%\newcommand{\Atrans}{\mathcal{A^{IOP\to ILC}}}
%\newcommand{\Atranso}{\Atrans^{\AIOP}}
\newcommand{\andc}{~\wedge~} % Can set to \wedge or \mbox{~and~}
\newcommand{\bigO}{\mathcal{O}}


\newcommand{\hamdist}{\ensuremath{\mathsf{hd}}} % hamming distance. Should it be in \mathrm to avoid italic?
\newcommand{\minhamdist}{\ensuremath{\mathsf{hd}}_{\mathrm{min}}} % hamming distance. Should it be in \mathrm to avoid italic?
\newcommand{\relham}{\mathsf{rhd}}
\newcommand{\hamw}{\ensuremath{\mathsf{wt}}} %hamming weight
\newcommand{\hamweight}{\hamw}

\newcommand{\proof}{\emph{Proof}}

\newcommand{\oc}{\mathrm{oc}}%opening complexity
\newcommand{\qc}{\mathrm{qc}}%query complexity
\newcommand{\qrc}{\mathrm{qrc}}%query round complexity (=1 in our paper)
\newcommand{\vc}{\mathrm{vcm}}%verifier communication
\newcommand{\vcm}{\vc}%alternative command for verifier communication
\newcommand{\vcp}{\mathrm{vcp}}%verifier computation
\newcommand{\pcm}{\mathrm{pcm}}%prover communication
\newcommand{\pcp}{\mathrm{pcp}}%prover computation
\newcommand{\viewV}{\mathsf{view}_{\V}}
\newcommand{\poly}{\textrm{poly}}
\newcommand{\numpack}{\mathrm{\#p}}
\newcommand{\tqs}{\mathrm{tqs}}

\newcommand{\DDelta}{d}

\newcommand{\transp}{{^\intercal}}

\newcommand{\polylog}{\ensuremath{\textrm{polylog}}}

%variables used in definition of ILC, IOP and compilation ILC->IOP


\newcommand{\numcol}{\nu}%number of columns used in a matrix in IOP section
\newcommand{\numround}{\mu}%total number of rounds in a protocol
\newcommand{\roundnum}{i}%refers to a specific round number
\newcommand{\eccr}{\vec{e}}%notation for vectors that are codeword
\newcommand{\eccmal}{\vec{e}^*}%notation for vectors the might be codewords but are committed to by a malicious prover
\newcommand{\sizeecc}{2\sizeeccrand}%size of codewords
\newcommand{\setsizeecc}{[\sizeecc]}%set corresponding to length of ecc. Set set from which challenge j are taken
\newcommand{\ECC}{\tilde{\mathsf{E}}_\codeset}%ERE error correcting code
\newcommand{\preECC}{\mathsf{E}_\codeset}% error correcting code used for the \ECC
\newcommand{\sizevect}{k}%the length of the vectors committed to in the first layer
%\newcommand{\sizevecr}{\sizevect}%only relevant if not all vectors have the same length
\newcommand{\coefvec}[1]{\vec{\gamma}_{{#1}}}%the coefficient used when opening linear combinations of committed vectors

%variables used in appendix
\newcommand{\app}{Appendix}
\newcommand{\totalnumgates}{N}%total number of gates in the arithmetic circuit
\newcommand{\numgates}{m}%number of addition and multiplication gates
\newcommand{\numcomp}{\mathfrak{m}}%number of vectors to be compressed using `binary-expansion' challenges
\newcommand{\numsqrt}{\mathfrak{n}}%number of vectors that simple sqrt argument is applied to
\newcommand{\lognumcomp}{\mu}%log of \numcomp

\newcommand{\event}{Err}
\newcommand{\eventsz}{F}
\newcommand{\eventszz}{G}

\newcommand{\vect}{{\vec{v}}}
\newcommand{\vectmal}[1]{\vect_{(#1)}^*}
\newcommand{\vecr}[1]{{\vect_{#1}}}
\newcommand{\numvect}[1]{{t_{#1}}}%total number of vectors committed to by round #1.
\newcommand{\totalnumvec}{\numvect{}}%total umber of vectors committed to
\newcommand{\Transcript}[1]{\text{Tr}(#1)}
%\newcommand{\View}[2]{\text{View}_{#1}(#2)}

\newcommand{\eccrand}{\vec{r}}
\newcommand{\eccrandmal}[1]{\eccrand_{(#1)}^*}
\newcommand{\sizeeccrand}{n}

\newcommand{\ecc}{\vec{e}}%notation for vectors that are codeword
\newcommand{\chalJ}{J}%set of challenges
\newcommand{\Chal}{Q}%challenge matrix in last round of ILC to IOP
\newcommand{\chalj}{j}%challenge values
\newcommand{\chals}{\sep}% number of challenges s
\newcommand{\chalx}{{\vec{\gamma}}}% values for the extra linear challenge
\newcommand{\chalxt}{\chalx}%Sune: think it is the same as \chalx, but not confident enough to do replace all.
\newcommand{\chaloptx}{\chalxt^*}% worst values for linear challenge


\newcommand{\claimr}{r}
\newcommand{\codeword}{c}
\newcommand{\restword}{v}

\newcommand{\indupara}{\kappa}

\newcommand{\codeset}{\mathcal{C}}
\newcommand{\maxdist}{\frac{\minhamdist}{3}}
\newcommand{\relmindist}{\mathsf{rhd}_{\mathrm{min}}}

%%The following 3 newcommands are using in the appendix appclaim

\newcommand{\xce}{{\eccmal_0}+\chalx E^*}
\newcommand{\xcer}{\vec{e}^*_0+(\chalx+r\chalx^*) E^*}
\newcommand{\xceo}{\eccmal_0+x^* E^*}


% Vectors

\newcommand{\HKSamp}{\mathsf{HK}} % Hash key sampling 
\newcommand{\HH}{\mathsf{H}} % Hash function
\newcommand{\HHFam}{\mathcal{H}}  % Hash family
\newcommand{\KeySpace}{\mathcal{K}}   % General Key Space
\newcommand{\DSpace}{\mathcal{D}}   % Domain
\newcommand{\RSpace}{\mathcal{R}}   % Range
\newcommand{\inpoly}{\alpha}   % input size polynomial
\newcommand{\outpoly}{\beta}   % Output size polynomial
\newcommand{\keypoly}{\kappa}   % Key size polynomial
\newcommand{\blocklen}{\kappa} % committed blocksize


%Compilation environment

\newtheorem{compilation}{Compilation}




%=-============================
% Primitives 
%\newcommand{\COM}{\mathcal{COM}} % Commitment Scheme

\newcommand{\heading}[1]{\vspace{.1cm}\noindent{\sc{\textsc{#1}}}.}
%title for subsubsections which reduce vertical space consumed
 
%======================================
% Packages


\usepackage{verbatim}
\usepackage{amsmath}
\usepackage{amssymb}
\usepackage{multirow}
\usepackage{color}
\usepackage{url}
\usepackage{ifthen}
%\usepackage{eufrak}
\usepackage{setspace}
\usepackage{threeparttable}
\usepackage{enumerate}
%\usepackage{arydshln}
\usepackage{centernot}
\usepackage{tabularx}
\usepackage{hhline}
\usepackage{booktabs} 
\usepackage{footmisc}
\usepackage{subfigure}
\usepackage{setspace}
\usepackage{stmaryrd}
\usepackage[latin1]{inputenc}
\usepackage{fancybox}
\usepackage{tikz}
\usepackage{tabularx}
\usepackage{listings}
\usepackage{mathtools}
\usepackage{etextools}
\usepackage{caption}

%\usepackage[ruled,noend]{algorithm2e}
\usepackage{algorithm}
 %package for algorithms and pseudocode
%\usepackage[noend]{algorithm2e} %package for algorithms and pseudocode

%\usepackage[backend=bibtex,firstinits=true]{biblatex}

\definecolor{almond}{rgb}{0.94, 0.87, 0.8} % The color used in the merkle tree

\DeclarePairedDelimiter\ceil{\lceil}{\rceil}
\DeclarePairedDelimiter\floor{\lfloor}{\rfloor}